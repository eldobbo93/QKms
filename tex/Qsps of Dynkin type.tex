\documentclass[a4 paper, 10pt]{article}

\usepackage{../mystyle}

\begin{document}
\title{An explicit construction for the quasi $K$-matrix}
\date{}
\maketitle

%\section{Preliminaries}

%Let $\fg$ be a complex semisimple lie algebra with Cartan subalgebra $\fh$. Let $\Phi \subset \fk^{\star}$ denote the corresponding root system. Associated to $\fg$ are the set of simple roots $\Pi = \{ \alpha_i \: | \: i \in I \}$ and the set of simple coroots $\Pi^{\vee} = \{h_i \: | \: i \in I \}$. 

%Denote by $W$ the Weyl group of $\fg$ generated by the simple reflections $s_i \in \text{GL}(\fh^{\star})$ for $i \in I$. The simple reflection $s_i$ is defined by

%\begin{equation} \label{eq:simprefl}
	%s_i(\alpha) = \alpha = \alpha(h_i)\alpha_i.
%\end{equation}

The aim of this section is to construct an explicit expression for the quasi $K$-matrix $\Qkm$ in the cases where we take the identity diagram automorphism and $X = \emptyset$. Throughout, we'll assume that $s_i = 0$ for all $i \in I$. This assumption covers almost all examples of this type, with the exceptions coming from the Dynkin diagrams of type A with an odd number of nodes. 


\subsection{The rank one quasi $K$-matrix}

The next Proposition is the main tool we use for verifying that we have a quasi $K$-matrix in low rank.

\begin{proposition} \label{QKmskew}
	For all $i \in I$ the quasi $K$-matrix satisfies
		\begin{align}
			\ri{i}{\Qkm} &= (q_i - q_i^{-1}) \overline{c_i} \Qkm E_i, \label{eq:riQkm}\\
			\ir{i}{\Qkm} &= (q_i - q_i^{-1}) q_i^2c_iE_i \Qkm.  \label{eq:irQkm}
		\end{align}
\end{proposition} 

\begin{proof}
 The proof follows from \cite[Prop 3.1]{a-BK15}. In particular, we write $\Qkm = \sum_{\mu \in Q^+} \Qkm_{\mu} \in \widehat{U^+}$ and use \cite[Eqns 6.2, 6.3]{a-BK15} 
\begin{align}
	\ri{i}{\Qkm_{\mu}} 	&= -(q_i - q_i^{-1}) \Qkm_{\mu+\Theta(\alpha_i)-\alpha_i} \overline{c_iX_i}, \label{eq:riQkmcomp}\\
	\ir{i}{\Qkm_{\mu}}	&= -(q_i - q_i^{-1}) q^{-(\Theta(\alpha_i),\alpha_i)}c_iX_i \Qkm_{\mu+\Theta(\alpha_i)-\alpha_i}, \label{eq:irQkmcom} 
\end{align} 
  noting that $\Theta(\alpha_i) = -\alpha_i$ and $X_i = -E_i$. 
 
 Then
 	\begin{align*}
 		\ri{i}{\Qkm} 	&= \ri{i}{\sum_{\mu \in Q^+} \Qkm_{\mu} } \\
 						&= \sum_{\mu \in Q^+} \ri{i}{\Qkm_{\mu}} \\
 						&= -(q_i - q_i^{-1})\overline{c_i}\big(\sum_{\mu \in Q^+} \Qkm_{\mu - 2\alpha_i}\big) E_i \\
 						&= -(q_i - q_i^{-1})\overline{c_i} \Qkm E_i   
 	\end{align*}
 as required. The proof for $\ir{i}{\Qkm}$ is analagous.
\end{proof}

\begin{remark}
	In fact, the relations \eqref{eq:riQkm} and \eqref{eq:irQkm} are equivalent to \eqref{eq:riQkmcomp} and \eqref{eq:irQkmcom}. Then by using results of Balagovi\'{c} and Kolb \cite[Theorem 6.10]{a-BK15}, we see that $\Qkm$ is the unique element satisfying \eqref{eq:riQkm} and \eqref{eq:irQkm}.
\end{remark}

For the remainder of this section, we will only consider the skew derivations $_{i}r$, so that we avoid dealing with the bar involution. 

Our first step is to construct the quasi $K$-matrix in the rank one case. Recall that $[n]_i$ denotes the $q_i$-binomial coefficient $\qbin{n}{1}_i $ \cite[1.3.3]{b-Lusztig94}. Let $\{ n \}_i := q_i^{n-1}[n]_i$ for $n \geq 1$ and $\{ 0 \}_i := 1$. Also, let $\{n\}_i!!$ denote the double factorial of ${n}_i$. 

\begin{lemma} \label{rk1Qkm}
	Let $\Qkm = \Qkm_{1}$ denote the quasi $K$-matrix of rank one. Then
		
		\begin{equation} \label{eq:rk1Qkm}
			\Qkm = \sum_{n} \dfrac{(q_i - q_i^{-1})^n(q_i^2c_i)^n}{ \{2n\}_{i}!! } E_1^{2n}.
		\end{equation}
\end{lemma}

\begin{proof}
	We prove this by direct calculation, showing that $\ir{1}{\Qkm} = (q_1 - q_1^{-1})q_1^2c_1 E_1 \Qkm$.
	By \cite[8.26]{b-Jantzen96}, we see that for all integers $m>0$, 
		\begin{equation} \label{eq: 1rE1}
			\ir{1}{E_1^m} = \{ m \}_1 E_1^m.
		\end{equation}
	Then
		\begin{align*}
			\ir{1}{\Qkm} 	&= \sum_{n} \dfrac{(q_1 - q_1^{-1})^n(q_1^2c_1)^n}{ \{2n\}_{1}!! } \ir{1}{E_1^{2n}} \\
							&= \sum_{n \geq 1} \dfrac{(q_1 - q_1^{-1})^n(q_1^2c_1)^n}{ \{2n\}_{1}!! } \{2n\}_1E_1^{2n-1} \\
							&= (q_1 - q_1^{-1})q_1^2c_1E_1 \sum_{n \geq 1} \dfrac{(q_1 - q_1^{-1})^{n-1}(q_1^2c_1)^{n-1}}{ \{2(n-1)\}_{1}!! }E_1^{2(n-1)} \\
							&= (q_1 - q_1^{-1})q_1^2c_1E_1 \Qkm
		\end{align*}				
	as required.
\end{proof}

\subsection{Partial quasi $K$-matrices}

We aim now for a construction for the quasi $K$-matrix of rank $n$. In order to do this, we require the following notation.

We denote by $\phi: \Ug^+ \rightarrow \Ug^+$ the algebra automorphism that takes $E_i$ to $q_i^2c_iE_1$. This extends to an algebra automorphism $\phi: \Ug \rightarrow \Ug$. For each $i \in I$, let $\Tconj{i}$ denote the conjugation of $\T{i}$ by $\phi$.

We let $\Qkm_{i}$ denote the rank one quasi $K$-matrix corresponding to the single node $i$. We introduce the following definition.


\begin{definition} \label{PartialK}
	Let $w = s_{i_1}s_{i_2} \dots s_{i_k}$ be a reduced expression in the Weyl group $W$. We define the partial quasi $K$-matrix $\Qkm_{w}$ by
	
	\begin{equation} \label{eq:PartialK}
		\Qkm_{w} := \Tconj{i_1 \dots i_{k-1}}(\Qkm_{i_k}) \dots \Tconj{i_1}(\Qkm_{i_2})\Qkm_{i_1}.
	\end{equation}
\end{definition}

Since the partial quasi $K$-matrix depends on choosing a reduced word $w \in W$, a useful property to have is that different expressions for the same reduced word give the same partial quasi $K$-matrix. This turns out to be true. 
\begin{theorem} \label{Qkmind}
	Let $w, w^{\prime} \in W$ be two reduced expressions for the same word. Then $\Qkm_{w} = \Qkm_{w^{\prime}}$. 
\end{theorem}

We postpone the proof until the end of the section, when we will have the necessary tools to tackle it.
The general strategy is to prove the result in the rank two cases $A1 \times A1$. $A2$, $B2$ and $G2$, then we can use this and the braid relations to prove the result more generally. The $A1 \times A1$ case is immediate since the two generators we obtain commute. 

Of interest is the case when we take $w$ to be a reduced expression for the longest word in the Weyl group. 
Then the following corollary gives explicit expressions for the quasi $K$-matrix.

\begin{corollary}
	Let $w_0 \in W$ be a reduced expression for the longest word. Then $\Qkm_{w_0} = \Qkm$, the quasi $K$-matrix.
\end{corollary}

\begin{proof}
	We may choose a reduced expression $w_0 = s_{i_1} \dots s_{i_k}$ such that $s_{i_1} \dots s_{i_{k-1}}(\alpha_{i_k}) = \alpha_j$ for some $j$ and $i_1 \neq j$. 
	Let $\tau: Q \rightarrow Q$ denote the involution on the roots that takes the root $\alpha_i$ to $\alpha_{n-i+1}$.
	Then since the longest word acts as $-$id on roots, we must have $s_{i_k} = s_{\tau(j)}$. 

	Now, $s_jw_0 = w_0s_{\tau(j)} = s_{i_1}s_{1_2}\dots s_{i_{k-1}}$ and hence by \cite[8.26, (4)]{b-Jantzen96}, it follows that 
	\[ \ir{j}{E_{i_1}} = \: \ir{j}{T_{i_1}(E_{i_2})} = \dots = \: \ir{j}{\T{i_1\dots i_{k-2}}(E_{i_{k-1}})} = 0.\]
	
	The longest word $w_0$ was chosen in such a way that
		\begin{equation}
			\Qkm_{w_0} = \Qkm_j \Tconj{i_1\dots i_{k-2}}(\Qkm_{i_{k-1}}) \dots \Tconj{i_1}(\Qkm_{i_2})\Qkm_{i_1}.
		\end{equation}
	Hence 
		\begin{align*}
			\ir{j}{\Qkm_{w_0}} 	&= \: \ir{j}{\Qkm_j} \Tconj{i_1 \dots i_{k-2}}(\Qkm_{i_{k-1}}) \dots \Tconj{i_1}(\Qkm_{i_2}) \Qkm_{i_1}\\
								&= (q_j - q_j^{-1})q_j^2c_jE_j\Qkm_{w_0}
		\end{align*}		
	by the same rank one calculation as Lemma \ref{rk1Qkm}. By Theorem \ref{Qkmind}, it follows that 
	
		\begin{equation}
			\ir{j}{\Qkm_{w_0}} = (q_j - q_j^{-1})q_j^2c_jE_j\Qkm_{w_0}
		\end{equation}		 
	independently of how we chose the longest word $w_0$. Since the index $j$ was chosen arbitrarily, this holds for every $j = 1, \dots, n$. Then by Proposition \ref{QKmskew}, the proof is complete.
\end{proof}


\subsection{The quasi $K$-matrix for type $A2$}
In $A2$, if $w, w^{\prime}$ are distinct reduced expressions for the same word, then the length of both must be at least three. In fact, since $s_1s_2s_1$ is an expression for the longest word, then the length of $w$ and $w^{\prime}$ must be three. Since the only relation in the Weyl group is $s_1s_2s_1 = s_2s_1s_2$, we show the following. 

\begin{proposition} \label{rk2A2}
	In the case of $A2$, we have $\Qkm_{s_1s_2s_1} = \Qkm_{s_2s_1s_2}$.
\end{proposition}

Before we prove this, we need to know how the Lusztig skew derivations $\ir{1}{}$ and $\ir{2}{}$ act on certain elements and their powers, and also some commutation relations. These are given in the following two lemmas, whose proofs are given by straightforward computation. We note that due to the symmetry in type $A2$, we drop the index from the notation $q_i$ and $\{ n \}_i$ for the remainder of this case.

\begin{lemma} \label{A2comm}
	We have
		\begin{align*}
			E_2^n E_1 &= q^n E_1 E_2^n - q\{ n \} E_2^{n-1} \T{1}(E_2),\\
			E_1^n E_2 &= q^n E_2 E_1^n - q\{ n \} E_1^{n-1} \T{2}(E_1),\\
			\T{1}(E_2)^n E_1 &= q^{-n} E_1 \T{1}(E_2)^n,\\
			\T{2}(E_1)^n E_2 &= q^{-n} E_2 \T{2}(E_1)^n,\\
		\end{align*}
\end{lemma}

\begin{lemma} \label{A2skew}
	We have
		\begin{align*}
			&\ir{ 1 }{ E_2^n } = \: \ir{ 1 }{ \T{2}(E_1)^n } = \: \ir{ 2 }{ E_1^n } = \: \ir{2}{ \T{1}(E_2)^n } = 0,\\
			&\ir{ 1 }{ E_1^n } = \{ n \} E_1^{n-1},\\
			&\ir{ 2 }{ E_2^n } = \{ n \} E_2^{n-1},\\
			&\ir{ 1 }{ \T{1}(E_2)^n } = (1 - q^{-2}) \{ n \} E_2 \T{1}(E_2)^{n-1},\\
			&\ir{ 2 }{ \T{2}(E_1)^n } = (1 - q^{-2}) \{ n \} E_1 \T{2}(E_1)^{n-1}.
		\end{align*}
\end{lemma}
\begin{proof}[Proof of Proposition \ref{rk2A2}]
	Our strategy is to show that both $\Qkm_{s_1s_2s_1}$ and $\Qkm_{s_2s_1s_2}$ are in fact expressions for the quasi $K$-matrix $\Qkm$. Then the uniqueness of such an element \cite[Theorem 6.10]{a-BK15} immediately implies that these two expressions are identical.
	
	Consider first the element $\Qkm_{s_1s_2s_1}$. By definition,
		
		\begin{equation}
			\Qkm_{s_1s_2s_1} = \Big(\sum_{n_1} \alpha_{n_1} E_2^{2n_1} \Big) \Big(\sum_{n_2} \beta_{n_2} \T{1}(E_2)^{2n_2} \Big) \Big(\sum_{n_3} \gamma_{n_3} E_1^{2n_3} \Big)
		\end{equation}
	
	where
		\begin{align*}
			\alpha_{n_1} &= \dfrac{(q-q^{-1})^{n_1}(q^2c_2)^{n_1}}{ \{ 2n_1 \}!!},\\
			\beta_{n_2} &= \dfrac{(q-q^{-1})^{n_2}(q^4c_1c_2)^{n_2}}{ \{ 2n_2 \}!!},\\
			\gamma_{n_3} &= \dfrac{(q-q^{-1})^{n_3}(q^2c_1)^{n_3}}{ \{ 2n_3 \}!!}.
		\end{align*}
		
	By Lemma \ref{A2skew}, we see that $\ir{2}{\Qkm_{s_1s_2s_1}} = (q-q^{-1})q^2c_2E_2\Qkm_{s_1s_2s_1}$ by a rank one calculation. We are left with calculating $\ir{1}{\Qkm_{s_1s_2s_1}}$. By the recursive formula
		\begin{equation}
			\ir{i}{xy} = \: \ir{i}{x}y + K_ixK_i^{-1} \:\ir{i}{y}
		\end{equation}

	and again by Lemma \ref{A2skew}, we see that
		\begin{align*}
			\ir{1}{\Qkm_{s_1s_2s_1}} 	&= \Big( \sum_{n_1} \alpha_{n_1} q^{-2n_1} E_2^{2n_1} \Big) \irr{1}{  \Big(\sum_{n_2} \beta_{n_2} \T{1}(E_2)^{2n_2} \Big) \Big(\sum_{n_3} \gamma_{n_3} E_1^{2n_3} \Big) } \\
										&= (1-q^{-2}) \Big( \sum_{n_1} \alpha_{n_1} q^{-2n_1} E_2^{2n_1} \Big) E_2\T{1}(E_2) \Big( \sum_{n_2 } \beta_{n_2 + 1} \{ 2(n_2+1) \} \T{1}(E_2)^{2n_2} \Big) \Big( \sum_{n_3} \gamma_{n_3} E_1^{2n_3} \Big) \\
										&\qquad{} +  \Big( \sum_{n_1} \alpha_{n_1} q^{-2n_1} E_2^{2n_1} \Big) \Big(\sum_{n_2} \beta_{n_2} q^{2n_2} \T{1}(E_2)^{2n_2} \Big) E_1 \Big( \sum_{n_3} \gamma_{n_3+1} \{ 2(n_3+1) \} E_1^{2n_3} \Big).
		\end{align*}
	
	Our aim is to show that $\ir{1}{\Qkm_{s_1s_2s_1} } = (q-q^{-1})q^2c_1E_1\Qkm_{s_1s_2s_1}$, so we use our commutation 	relations from Lemma \ref{A2comm} on the second summand. Since
		\begin{equation}
			\gamma_{n_3+1} \{2(n_3 + 1)\} = (q-q^{-1})q^2c_1 \gamma_{n_3},
		\end{equation}
	it is easy to see that we will obtain $(q - q^{-1})q^2c_1E_1\Qkm_{s_1s_2s_1}$ with an additional error term from commuting powers of $E_2$ with $E_1$. This error term is given by
	
		\begin{equation*}
			-q \Big( \sum_{n_1} \alpha_{n_1 + 1} \{ 2(n_1 + 1) \} q^{-2(n_1 + 1)} E_2^{2n_1} \Big) E_2\T{1}(E_2) \Big(\sum_{n_2} \beta_{n_2}  \T{1}(E_2)^{2n_2} \Big) E_1 \Big( \sum_{n_3} \gamma_{n_3+1} \{ 2(n_3+1) \} E_1^{2n_3} \Big).
		\end{equation*}
	We compare this with the first summand of $\ir{1}{\Qkm_{s_1s_2s_1}}$ above. We do this by comparing coefficients of $E_2^{2n_1 + 1}\T{1}(E_2)^{2n_2+1}E_1^{2n_3}$. We have
	
		\begin{align*}
			(1 - q^{-2})q^{-2n_1}\alpha_{n_1}\beta_{n_2+1}\{2(n_2+1)\}\gamma_{n_3} &= q^{-1}(q-q^{-1})^2(q^4c_1c_2)q^{-2n_1}\alpha_{n_1}\beta_{n_2}\gamma_{n_3} \\
				&= q (q-q^{-1})(q^2c_2)\alpha_{n_1}q^{-2(n_1+1)}\beta_{n_2}(q-q^{-1})(q^2c_1)\gamma_{n_3} \\
				&= q \alpha_{n_1 + 1} \{ 2( n_1 + 1 ) \} q^{ -2 (n_1 + 1) } \beta_{n_2} \gamma_{ n_3+1 } \{ 2( n_3 + 1 ) \}
		\end{align*}
		
Hence the error term cancels with the first summand. 

We conclude that $\ir{1}{\Qkm_{s_1s_2s_1}} = (q-q^{-1})q^2c_1E_1\Qkm_{s_1s_2s_1}$ and hence $\Qkm_{s_1s_2s_1}$ is indeed an expression for the quasi $K$-matrix. 

Instead of repeating the same calculations for $\Qkm_{s_2s_1s_2}$, we rely on the underlying symmetry in type $A2$, which allows us to conclude that 

	\begin{equation}
		\Qkm_{s_2s_1s_2} = \Big(\sum_{n_1} \alpha_{n_1}^{\prime} E_1^{2n_1} \Big) \Big(\sum_{n_2} \beta_{n_2}^{\prime} \T{2}(E_1)^{2n_2} \Big) \Big(\sum_{n_3} \gamma_{n_3}^{\prime} E_2^{2n_3} \Big)
	\end{equation}

is also an expression for the quasi $K$-matrix, which completes the proof.
\end{proof}

\subsection{The quasi $K$-matrix for type $B2$.}
We now move to the type $B2$ case. We have the following Dynkin diagram where the second node corresponds to the short root. 

\begin{center}
\includegraphics[scale=1]{../img/B2}
\end{center}

The strategy remains largely the same, where we use the two expressions for the longest word $w = s_1s_2s_1s_2$ and $w^{\prime} = s_2s_1s_2s_1$. As one might expect, the computations are more complex here, and since there is no underlying symmetry of the Dynkin diagram of type $B2$, we will have to explicitly show that both $\Qkm_w$ and $\Qkm_{w^{\prime}}$ are expressions for the quasi $K$-matrix. Our partial quasi $K$-matrices take the form

	\begin{equation} \label{eq:B2Qkm1}
		\Qkm_{w} = \Big( \sum_{n_1} \alpha_{n_1} E_2^{2n_1} \Big)	 \Big( \sum_{n_2} \beta_{n_2} \T{12}(E_1)^{2n_2} \Big) 
			\Big( \sum_{n_3} \gamma_{n_3} \T{1}(E_2)^{2n_3} \Big) 	\Big( \sum_{n_4} \delta_{n_4} E_1^{2n_4} \Big),
	\end{equation}
	\vspace{10pt}
	\begin{equation} \label{eq:B2Qkm2}
		\Qkm_{w^{\prime}} = \Big( \sum_{n_1} \alpha_{n_1}^{\prime} E_1^{2n_1} \Big)	 \Big( \sum_{n_2} \beta_{n_2}^{\prime} \T{21}(E_2)^{2n_2} \Big) 
			\Big( \sum_{n_3} \gamma_{n_3}^{\prime} \T{2}(E_1)^{2n_3} \Big) 	\Big( \sum_{n_4} \delta_{n_4}^{\prime} E_2^{2n_4} \Big).
	\end{equation}
	\vspace{5pt}
	
We will again need to know some commutation relations and how the Lusztig skew derivations act on powers of elements. These are given in the following two lemmas.

\begin{lemma} \label{B2comm}
	We have
		\begin{align*}
			E_2^{ n } E_1  			&=  q^{ 2n } E_1 E_2^{ n } - q^{ 2 } \{ n \}_{2} \T{1}(E_2) E_2^{ n-1 } + q \{ n-1 \}_{2} \{ n \}_2 E_2^{n-2} \T{12}(E_1), \\
			E_1^{ n } E_2			&=	q^{ 2n } E_2 E_1^{ n } - q^{ 2 } \{ n \}_{1} E_1^{ n-1 } \T{21}(E_2), \\
			\T{1}(E_2)^{ n } E_1		&=	q^{ -2n } E_1 \T{1}(E_2)^{ n }, \\
			\T{2}(E_1)^{ n } E_2		&=	q^{ -2n } E_2 \T{2}(E_1)^{ n }, \\
			\T{1}(E_2)^{ n } E_2		&=	E_2 \T{1}(E_2)^{ n } + [ 2 ]_{2} \{ n \}_{2} \T{12}(E_1) \T{1}(E_2)^{ n-1 }, \\
			\T{1}(E_2) E_2^{ n }		&=	E_2^{ n } \T{1}(E_2) + [ 2 ]_{2} \{ n \}_{2} E_2^{ n-1 } \T{12}(E_1), \\
			\T{12}(E_1)^{ n } E_1	&=	E_1 \T{12}(E_1)^{ n } - \frac{(q^2 - 1)}{ [ 2 ]_{2} } \{ n \}_{1} \T{12}(E_1)^{ n-1 } \T{1}(E_2)^{ 2 }, \\
			\T{21}(E_2)^{ n } E_2	&=	E_2 \T{21}(E_2)^{ n } - [ 2 ]_{2} \{ n \}_{2} \T{21}(E_2)^{ n-1 } \T{2}(E_1).
		\end{align*}
\end{lemma}

\begin{lemma} \label{B2skew}
	We have
		\begin{align*}
			\ir{ 1 }{ \T{21}(E_2)^{ n } } 	&=	\: \ir{ 1 }{ \T{2}(E_1)^{ n } } = \: \ir{ 1 }{ E_2^{ n } } = 0, \\
			\ir{ 2 }{ \T{12}(E_1)^{ n } } 	&=	\: \ir{ 2 }{ \T{1}(E_2)^{ n } } = \: \ir{ 2 }{ E_1^{ n } } = 0, \\
			\ir{ 1 }{ E_1^{ n } }			&=	\{ n \}_{1} E_1^{ n-1 }, \\
			\ir{ 1 }{ \T{1}(E_2)^{ n } }		&=	q^{ -2 }( q^{ 2 } - q^{ -2 } ) \{ n \}_{2} E_2 \T{1}(E_2)^{ n-1 } + q^{ -1 }( q^{ 2 } - q^{ -2 } ) \{ n \}_{2} \{ n-1 \}_{2} \T{12}(E_1) \T{1}(E_2)^{ n-2 }, \\
			\ir{ 1 }{ \T{12}(E_1)^{ n } }	&=	q^{ -3 }( q - q^{ -1 } )^{ 2 } \{ n \}_{1} E_2^{ 2 } \T{12}(E_1)^{ n-1 }, \\
			\ir{ 2 }{ E_2^{ n } }			&=	\{ n \}_{2} E_2^{ n-1 }, \\
			\ir{ 2 }{ \T{2}(E_1)^{ n } }		&=	( q - q^{ -1 } ) \{ n \}_{1} \T{21}(E_2) \T{2}(E_1)^{ n-1 } , \\
			\ir{ 2 }{ \T{21}(E_2)^{ n } }	&=	q^{ -2 }( q^{ 2 } - q^{ -2 } ) \{ n \}_{2} E_1 \T{21}(E_2)^{ n-1 }.
		\end{align*}
\end{lemma}

\begin{proposition}
 The partial quasi $K$-matrix given in \eqref{eq:B2Qkm1} is an expression for the quasi $K$-matrix $\Qkm$.
\end{proposition}

\begin{proof}

We write $\Qkm_{w} = \Y_{1}\Y_{2}\Y_{3}\Y_{4}$ where each $\Y_{i}$ corresponds to the $i$-th factor in \eqref{eq:B2Qkm1}. By Lemma \ref{B2skew} and Proposition \ref{rk1Qkm}, we immediately see that 
	\begin{equation}
		\ir{2}{\Qkm_{w}} = (q - q^{-1})q^2c_2 E_2 \Qkm_{w}.
	\end{equation}

We now want to show that
	\begin{equation}
		\ir{1}{\Qkm_{w}} = (q^2 - q^{-2})q^4c_1 E_1 \Qkm_{w}.
	\end{equation}

For each $i = 1, 2, 3, 4,$ let $\Yp_{i} = K_1 \Y_{i} K_1^{-1}$. Note that since $\T{12}(E_1)$ has corresponding root $\alpha_1 + 2\alpha_2$, we have $\Y_{2} = \Yp_{2}$. Then we see that
	\begin{equation}
		\ir{1}{\Qkm_{w}} = \Yp_{1} \: \ir{1}{\Y_{2}} \Y_{3} \Y_{4} + \Yp_{1} \Yp_{2} \: \ir{1}{\Y_{3}} \Y_{4} + \Yp_{1} \Yp_{2} \Yp_{3} \: \ir{1}{\Y_{4}}
	\end{equation}

We use Lemma \eqref{B2skew} and the fact that the coefficients $\alpha_{n_1}, \beta_{n_2}, \gamma_{n_3}, \delta_{n_4}$ satisfy a recurrence to write down expressions for $\ir{1}{\Y_2}, \: \ir{1}{\Y_3} \text{and} \: \ir{1}{\Y_4}$. 
	\begin{align*}
		\ir{1}{\Y_2} &= q^{ -3 } ( q^{ 2 } - q^{ -2 } ) ( q - q^{ -1 } )^{ 2 } ( q^{ 4 } c_1 ) ( q^{ 4 } c_2^{ 2 } ) E_2^{ 2 } \Y_{2} \T{12}(E_1), \\ 	
		\ir{1}{\Y_3} 	&= q^{ -2 } ( q^{ 2 } - q^{ -2 } ) ( q - q^{ -1 } ) ( q^{ 4 } c_1 ) ( q^{ 2 } c_2 ) E_2 \Y_{3} \T{1}(E_2) \\
						&\qquad{} + q^{ -1 } ( q^{ 2 } - q^{ -2 } )( q - q^{ -1 } ) ( q^{ 4 } c_1 ) ( q^{ 2 } c_2 ) \T{12}(E_1) \Big( \sum_{n_3} \gamma_{n_3} \{ 2n_3 + 1 \}_{2} \T{1}(E_2)^{2n_3} \Big), \\
		\ir{1}{\Y_4}		&= ( q^{ 2 } - q^{ -2 } ) ( q^{ 4 } c_1 ) E_1 \Y_{4}.		
	\end{align*}
	
The expression for $ \ir{1}{\Y_3}$ we can split further using the fact that for $n_3 \geq 1$, \[ \{ 2n_3 + 1 \}_{2} = 1 + q^{ 2 }\{ 2n_3 \}_{2}. \]
Hence,
	\begin{align*}
		\sum_{n_3} \gamma_{n_3} \{ 2n_3 + 1 \}_{2} \T{1}(E_2)^{2n_3} = \Y_{3} + q^{ 2 } ( q - q^{ -1 } ) ( q^{ 4 } c_1 ) ( q^{ 2 } c_2 ) \Y_{3} \T{1}(E_2)^{ 2 }.
	\end{align*}

So we may write $\ir{1}{\Y_{3}} = \mathcal{Z}_1 + \mathcal{Z}_2 + \mathcal{Z}_3$ where
	\begin{align*}
		\mathcal{Z}_1 &= q^{ -2 } ( q^{ 2 } - q^{ -2 } ) ( q - q^{ -1 } ) ( q^{ 4 } c_1 ) ( q^{ 2 } c_2 ) E_2 \Y_{3} \T{1}(E_2), \\
		\mathcal{Z}_2 &= q^{ -1 } ( q^{ 2 } - q^{ -2 } )( q - q^{ -1 } ) ( q^{ 4 } c_1 ) ( q^{ 2 } c_2 ) \T{12}(E_1) \Y_{3}, \\
		\mathcal{Z}_3 &= q ( q^{ 2 } - q^{ -2 } )( q - q^{ -1 } )^{ 2 } ( q^{ 8 } c_1^ { 2 } ) ( q^{ 4 } c_2^{ 2 } ) \T{12}(E_1) \Y_{3} \T{1}(E_2)^{ 2 }.
	\end{align*}
	
Using our commutation relations from Lemma \ref{B2comm}, we look at the term $ \Yp_{1} \Yp_{2} \Yp_{3} \: \ir{1}{\Y_4}$ in more detail. We have
	\begin{align*}
		\Yp_{1} \Yp_{2} \Yp_{3} \: \ir{1}{\Y_4} 	&= ( q^{ 2 } - q^{ -2 } )( q^{ 4 } c_1 ) \Yp_{1} \Yp_{2} \Yp_{3} E_1 \Y_{4} \\
												&= ( q^{ 2 } - q^{ -2 } )( q^{ 4 } c_1 ) \Yp_{1} \Yp_{2} E_1 \Y_{3} \Y_{4} \\
												&= ( q^{ 2 } - q^{ -2 } )( q^{ 4 } c_1 ) \Yp_{1} \Big( E_1 \Y_2 - q( q - q^{ -1 } )^{ 2 } ( q^{ 4 } c_1 ) ( q^{ 4 } c_2^{ 2 } ) \Yp_{2} \T{12}(E_1)\T{1}(E_2)^{ 2 } \Big) \Y_{3} \Y_{4} \\
												&= ( q^{ 2 } - q^{ -2 } )( q^{ 4 } c_1 ) \Yp_{1} E_1 \Y_{2} \Y_{3} \Y_{4} - \Yp_{1} \Yp_{2} \mathcal{Z}_3 \Y_{4}.
	\end{align*} 

Now, we commute $\Yp_{1}$ with $E_1$ to obtain
	\begin{align}
		( q^{ 2 } - q^{ -2 } )( q^{ 4 } c_1 ) \Yp_{1} E_1 \Y_{2} \Y_{3} \Y_{4} 
			&= ( q^{ 2 } - q^{ -2 } )( q^{ 4 } c_1 ) \Big( E_1 \Y_{1} - q^{ -2 } ( q - q^{ -1 } ) ( q^{ 2 } c_2 ) \T{1}(E_2) \Yp_{1} E_2 \nonumber \\
		 	&\quad{} +  q^{-3} ( q - q^{ -1 } )( q^{ 2} c_2 ) \big( \sum_{n_1} \alpha_{n_1} \{ 2n_1 + 1 \}_2 q^{-4n_1} E_2^{2n_1} \big) \T{12}(E_1) \Big) \Y_{2} \Y_{3} \Y_{4} \label{eq:Y1E1comm} 
	\end{align}
	
Using the commutation relation between $\T{1}(E_2)$ and powers of $E_2$, we can write
	\begin{align*}
		\T{1}(E_2) \Yp_{1} E_2 &= \Yp_{1} E_2 \T{1}(E_2) + [ 2 ]_{2} \Yp_{1} \T{12}(E_1) + q^{ -2 } [ 2 ]_{2} ( q - q^{ -1 } ) ( q^{ 2 } c_2 ) \Yp_{1} E_2^{ 2 } \T{12}(E_1).
	\end{align*}

We can also rewrite
	\begin{align*}
		\sum_{n_1} \alpha_{n_1} \{ 2n_1 + 1 \}_2 q^{-4n_1} E_2^{2n_1} &= \Yp_{1} + q^{ -2 }( q - q^{ -1 } ) ( q^{ 2 } c_ 2 ) \Yp_{1} E_2^{ 2 }.
	\end{align*}
	
Substituting these into \eqref{eq:Y1E1comm} and simplifying, we obtain
	\begin{align*}
		( q^{ 2 } - q^{ -2 } )( q^{ 4 } c_1 )& \Yp_{1} E_1 \Y_{2} \Y_{3} \Y_{4} \\
			&= ( q^{ 2 } - q^{ -2 } )( q^{ 4 } c_1 ) \Big( E_1 \Y_{1} - q^{ -2 }( q - q^{ -1 } ) ( q^{ 2 } c_2 ) \Yp_{1} E_2 \T{1}(E_2)\\ 
			&\quad{}- q^{ -1 } ( q - q^{ -1 } ) ( q^{ 2 } c_2 ) \Yp_{1} \T{12}(E_1) - q^{ -3 }( q - q^{ -1 } )^{ 2 } ( q^{ 4 } c_2^{ 2 } ) \Yp_{1} E_2^{ 2 } \T{12}(E_1) \Big) \Y_{2} \Y_{3} \Y_{4}\\
			&= ( q^{ 2 } - q^{ -2 } ) ( q^{ 4 } c_1 ) E_1 \Y_{1} \Y_{2} \Y_{3} \Y_{4} - \Yp_{1} \Y_{2} \mathcal{Z}_{1} \Y_{4} - \Yp_{1} \Y_{2} \mathcal{Z}_{2} \Y_{4} - \Yp_{1} \: \ir{1}{\Y_{2}} \Y_{3} \Y_{4}.
	\end{align*}

The term $\Yp_{1} \Y_{2} \mathcal{Z}_1 \Y_{4}$ has appeared here since $E_2\T{1}(E_2) \Y_{2} = \Y_{2} E_2\T{1}(E_2)$.

Hence, we have
	\begin{align*}
		\ir{1}{\Qkm_{w}} 
			&= \Yp_{1} \: \ir{1}{\Y_{2}} \Y_{3} \Y_{4} + \Yp_{1} \Yp_{2} \: \ir{1}{\Y_{3}} \Y_{4} + \Yp_{1} \Yp_{2} \Yp_{3} \: \ir{1}{\Y_{4}} \\
			&= \Yp_{1} \: \ir{1}{\Y_{2}} \Y_{3} \Y_{4} + \Yp_{1} \Yp_{2} ( \mathcal{Z}_1 + \mathcal{Z}_2 + \mathcal{Z}_3 ) \Y_{4} + ( q^{ 2 } - q^{ -2 } ) ( q^{ 4 } c_1 ) E_1 \Y_{1} \Y_{2} \Y_{3} \Y_{4}\\ 
			&\qquad{}- \Yp_{1} \Y_{2} \mathcal{Z}_{1} \Y_{4} - \Yp_{1} \Y_{2} \mathcal{Z}_{2} \Y_{4} - \Yp_{1} \: \ir{1}{\Y_{2}} \Y_{3} \Y_{4} - \Yp_{1} \Yp_{2} \mathcal{Z}_3 \Y_{4}\\
			&= ( q^{ 2 } - q^{ -2 } ) ( q^{ 4 } c_1 ) E_1 \Y_1 \Y_2 \Y_3 \Y_4 \\
			&= ( q^{ 2 } - q^{ -2 } ) ( q^{ 4 } c_1 ) E_1 \Qkm_{w}.
	\end{align*}
This completes the proof.

\end{proof}

\begin{proposition}
 The partial quasi $K$-matrix given in \eqref{eq:B2Qkm2} is an expression for the quasi $K$-matrix $\Qkm$. 
\end{proposition}

\begin{proof}
	We proceed in the same way as the previous proof. Let $\Qkm_{w^{\prime}} = \Y_{1} \Y_{2} \Y_{3} \Y_{4}$ where each $\Y_{i}$ corresponds to the $i$-th factor in \eqref{eq:B2Qkm2}. We see that 
	\begin{equation}
		\ir{1}{\Qkm_{w^{\prime}}} = ( q^{ 2 } - q^{ -2 } ) q^{ 4 } c_1 E_1 \Qkm_{w^{\prime}}.
	\end{equation}

We want to show that
	\begin{equation}
		\ir{2}{\Qkm_{w^{\prime}}} = ( q - q^{ -1 } ) q^{ 2 } c_2 E_2 \Qkm_{w^{\prime}}.
	\end{equation}
For each $i = 1, 2, 3, 4$, let $\Yp_{i} = K_2 \Y_{i} K_2^{-1}$. Since the element $\T{21}(E_2)$ has corresponding root  $\alpha_1 + \alpha_2$, we see that $\Yp_{2} = \Y_{2}$. Now, 
	\begin{equation}
		\ir{2}{\Qkm_{w^{\prime}}} = \Yp_{1} \: \ir{2}{\Y_{2}} \Y_{3} \Y_{4} + \Yp_{1} \Yp_{2} \: \ir{2}{\Y_{3}} \Y_{4} + \Yp_{1} \Yp_{2} \Yp_{3} \: \ir{2}{\Y_{4}}.
	\end{equation}

Using the expressions for the Luszig skew derivations given in Lemma \ref{B2skew}, we see that
	\begin{align*}
		\ir{2}{\Y_{2}}	&=	q^{ -2 }( q^{ 2 } - q^{ -2 } )( q - q^{ -1 } ) ( q^{ 4 } c_1 ) ( q^{ 2 } c_2 ) E_1 \Y_{2} \T{21}(E_2),\\
		\ir{2}{\Y_{3}}	&=	( q^{ 2 } - q^{ -2 } )( q - q^{ -1 } ) ( q^{ 4 } c_1 ) ( q^{ 4 } c_2^{ 2 } ) \T{21}(E_2) \Y_{3} \T{2}(E_1),\\
		\ir{2}{\Y_{4}}	&=	( q - q^{ -1 } ) ( q^{ 2 } c_2 ) E_2 \Y_{4}.
	\end{align*}
	
We look at the term $\Yp_{1} \Yp_{2} \Yp_{3} \: \ir{2}{\Y_{4}}$ in more detail and use the commutation relations from Lemma \ref{B2comm} to obtain
	\begin{align*}
		\Yp_{1} \Yp_{2} \Yp_{3} \: \ir{2}{\Y_{4}}	
			&=	( q - q^{ -1 } )( q^{ 2 } c_2 ) \Yp_{1} \Yp_{2} \Yp_{3} E_2 \Y_{4} \\
			&=	( q - q^{ -1 } )( q^{ 2 } c_2 ) \Yp_{1} \Yp_{2} E_2 \Y_{3} \Y_{4} \\
			&=	( q - q^{ -1 } )( q^{ 2 } c_2 ) \Yp_{1} \big( E_2 \Y_{2} - [ 2 ]_{2} ( q - q^{ -1 })( q^{ 4 } c_1)( q^{ 2 } c_2 ) \Y_{2} \T{21}(E_2) \T{2}(E_1) \big) \Y_{3} \Y_{4}\\
			&=	( q - q^{ -1 } )( q^{ 2 } c_2 ) \Yp_{1} E_2 \Y_{2} \Y_{3} \Y_{4} - \Yp_{1} \Y_{2} \: \ir{2}{\Y_{3}} \Y_{4}.
	\end{align*}
	
Now we commute $\Yp_{1}$ with $E_2$ to obtain
	\begin{align*}
		\Yp_{1} \Yp_{2} \Yp_{3} \: \ir{2}{\Y_{4}}
			&= ( q - q^{ -1 } )( q^{ 2 } c_2 ) \big( E_2 \Y_{1} - q^{ -2 }( q^{ 2 } - q^{ -2 } )( q^{ 4 } c_1 )\Yp_{1} E_1 \T{21}(E_2) \big) \Y_{2} \Y_{3} \Y_{4}\\
			&\qquad{} - \Yp_{1} \Yp_{2} \: \ir{2}{\Y_{3}} \Y_{4}\\
			&= ( q - q^{ -1 } )( q^{ 2 } c_2 ) E_2 \Y_{1} \Y_{2} \Y_{3} \Y_{4} - \Yp_{1} \: \ir{2}{\Y_{2}} \Y_{3} \Y_{4} - \Yp_{1} \Yp_{2} \: \ir{2}{\Y_{3}} \Y_{4}.
	\end{align*}
	
Hence we find that
	\begin{align*}
		\ir{2}{\Qkm_{w^{\prime}}} = ( q - q^{ -1 } )( q^{ 2 } c_2 ) E_2 \Qkm_{w^{\prime}}
	\end{align*}

as required. 
\end{proof}


\subsection{Proof of Theorem \ref{Qkmind}}
We are now in the position to prove Theorem \ref{Qkmind}. 

\begin{proof} Let $w$ and $w^{\prime}$ be two distinct expressions for the same word and suppose $w^{\prime}$ differs from $w$ by a single relation. The possible relations are

\begin{align*}
	s_is_j &= s_js_i\\
	s_is_{i+1}s_i &= s_{i+1}s_is_{i+1}\\
	s_is_{i+1}s_is_{i+1} &= s_{i+1}s_is_{i+1}s_i
\end{align*}

We consider the second relation; the same argument allows one to deal with the other relations. Let

\begin{align*}
 w &= s_{i_1} \dots s_{i_k} (s_is_{i+1}s_i) s_{i_{k+4}} \dots s_{i_l},\\
 w^{\prime} &= s_{i_1} \dots s_{i_k} (s_{i+1}s_{i}s_{i+1}) s_{i_{k+4}} \dots s_{i_l}.
\end{align*}

Then, we have

\begin{align*}
 \Qkm_w = T_{i_1 \dots i_{l-1}}(\Qkm_l) \dots T_{i_1 \dots i_k,i,i+1,i}(\Qkm_{i_{k+4}}) T_{i_1 \dots i_k}\Big(\Qkm_{s_is_{i+1}s_i} \Big) T_{i_1 \dots i_{k-1}}(\Qkm_{i_k}) \dots T_{i_1}(\Qkm_{i_2}) \Qkm_{i_1}.
\end{align*}

Now, $T_{i}T_{i+1}T_{i} = T_{i+1}T_{i}T_{i+1}$ and $\Qkm_{s_is_{i+1}s_i} = \Qkm_{s_{i+1}s_is_{i+1}}$ by Proposition \ref{rk2A2}. Hence we have

\begin{align*}
  \Qkm_w = T_{i_1 \dots i_{l-1}}(\Qkm_l) \dots T_{i_1 \dots i_k,i+1,i,i+1}(\Qkm_{i_{k+4}}) T_{i_1 \dots i_k}\Big( \Qkm_{s_{i+1}s_is_{i+1}} \Big) T_{i_1 \dots i_{k-1}}(\Qkm_{i_k}) \dots T_{i_1}(\Qkm_{i_2}) \Qkm_{i_1},
\end{align*} 

but this is exactly $\Qkm_{w^{\prime}}$.

If $w^{\prime}$ differs from $w$ by more than a single relation, then we obtain a sequence of distinct reduced expressions for the word $w$

\begin{align*}
 w = w_0, w_1, \dots w_m = w^{\prime}
\end{align*}

where for $j = 0, \dots m-1$, the word $w_j$ differs from $w_{j+1}$ by a single relation. By the above, we have

\begin{align*}
 \Qkm_w = \Qkm_{w_0} = \Qkm_{w_1} = \dots = \Qkm_{w_m} = \Qkm_{w^{\prime}},
\end{align*}
which completes the proof.

\end{proof}

\subsubsection*{To do}
\begin{itemize}
\item The calculation for $G2$.
\item Add this to $0.5$ via relation.
\item Turn Remark 1 into a Proposition (Possibly replace prev Prop by this)
\end{itemize}
%%%%%%%%%%%%%%%%%%%%%%%%%%%%%%%%%%%%%%%%%%%
\bibliographystyle{amsalpha}
\bibliography{../litbank}
%%%%%%%%%%%%%%%%%%%%%%%%%%%%%%%%%%%%%%%%%%%


\end{document} 
